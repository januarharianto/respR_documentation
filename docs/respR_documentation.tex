\documentclass[]{book}
\usepackage{lmodern}
\usepackage{amssymb,amsmath}
\usepackage{ifxetex,ifluatex}
\usepackage{fixltx2e} % provides \textsubscript
\ifnum 0\ifxetex 1\fi\ifluatex 1\fi=0 % if pdftex
  \usepackage[T1]{fontenc}
  \usepackage[utf8]{inputenc}
\else % if luatex or xelatex
  \ifxetex
    \usepackage{mathspec}
  \else
    \usepackage{fontspec}
  \fi
  \defaultfontfeatures{Ligatures=TeX,Scale=MatchLowercase}
\fi
% use upquote if available, for straight quotes in verbatim environments
\IfFileExists{upquote.sty}{\usepackage{upquote}}{}
% use microtype if available
\IfFileExists{microtype.sty}{%
\usepackage{microtype}
\UseMicrotypeSet[protrusion]{basicmath} % disable protrusion for tt fonts
}{}
\usepackage{hyperref}
\hypersetup{unicode=true,
            pdftitle={respR},
            pdfauthor={written by Januar Harianto and Nicholas Carey},
            pdfborder={0 0 0},
            breaklinks=true}
\urlstyle{same}  % don't use monospace font for urls
\usepackage{natbib}
\bibliographystyle{apalike}
\usepackage{color}
\usepackage{fancyvrb}
\newcommand{\VerbBar}{|}
\newcommand{\VERB}{\Verb[commandchars=\\\{\}]}
\DefineVerbatimEnvironment{Highlighting}{Verbatim}{commandchars=\\\{\}}
% Add ',fontsize=\small' for more characters per line
\usepackage{framed}
\definecolor{shadecolor}{RGB}{248,248,248}
\newenvironment{Shaded}{\begin{snugshade}}{\end{snugshade}}
\newcommand{\KeywordTok}[1]{\textcolor[rgb]{0.13,0.29,0.53}{\textbf{#1}}}
\newcommand{\DataTypeTok}[1]{\textcolor[rgb]{0.13,0.29,0.53}{#1}}
\newcommand{\DecValTok}[1]{\textcolor[rgb]{0.00,0.00,0.81}{#1}}
\newcommand{\BaseNTok}[1]{\textcolor[rgb]{0.00,0.00,0.81}{#1}}
\newcommand{\FloatTok}[1]{\textcolor[rgb]{0.00,0.00,0.81}{#1}}
\newcommand{\ConstantTok}[1]{\textcolor[rgb]{0.00,0.00,0.00}{#1}}
\newcommand{\CharTok}[1]{\textcolor[rgb]{0.31,0.60,0.02}{#1}}
\newcommand{\SpecialCharTok}[1]{\textcolor[rgb]{0.00,0.00,0.00}{#1}}
\newcommand{\StringTok}[1]{\textcolor[rgb]{0.31,0.60,0.02}{#1}}
\newcommand{\VerbatimStringTok}[1]{\textcolor[rgb]{0.31,0.60,0.02}{#1}}
\newcommand{\SpecialStringTok}[1]{\textcolor[rgb]{0.31,0.60,0.02}{#1}}
\newcommand{\ImportTok}[1]{#1}
\newcommand{\CommentTok}[1]{\textcolor[rgb]{0.56,0.35,0.01}{\textit{#1}}}
\newcommand{\DocumentationTok}[1]{\textcolor[rgb]{0.56,0.35,0.01}{\textbf{\textit{#1}}}}
\newcommand{\AnnotationTok}[1]{\textcolor[rgb]{0.56,0.35,0.01}{\textbf{\textit{#1}}}}
\newcommand{\CommentVarTok}[1]{\textcolor[rgb]{0.56,0.35,0.01}{\textbf{\textit{#1}}}}
\newcommand{\OtherTok}[1]{\textcolor[rgb]{0.56,0.35,0.01}{#1}}
\newcommand{\FunctionTok}[1]{\textcolor[rgb]{0.00,0.00,0.00}{#1}}
\newcommand{\VariableTok}[1]{\textcolor[rgb]{0.00,0.00,0.00}{#1}}
\newcommand{\ControlFlowTok}[1]{\textcolor[rgb]{0.13,0.29,0.53}{\textbf{#1}}}
\newcommand{\OperatorTok}[1]{\textcolor[rgb]{0.81,0.36,0.00}{\textbf{#1}}}
\newcommand{\BuiltInTok}[1]{#1}
\newcommand{\ExtensionTok}[1]{#1}
\newcommand{\PreprocessorTok}[1]{\textcolor[rgb]{0.56,0.35,0.01}{\textit{#1}}}
\newcommand{\AttributeTok}[1]{\textcolor[rgb]{0.77,0.63,0.00}{#1}}
\newcommand{\RegionMarkerTok}[1]{#1}
\newcommand{\InformationTok}[1]{\textcolor[rgb]{0.56,0.35,0.01}{\textbf{\textit{#1}}}}
\newcommand{\WarningTok}[1]{\textcolor[rgb]{0.56,0.35,0.01}{\textbf{\textit{#1}}}}
\newcommand{\AlertTok}[1]{\textcolor[rgb]{0.94,0.16,0.16}{#1}}
\newcommand{\ErrorTok}[1]{\textcolor[rgb]{0.64,0.00,0.00}{\textbf{#1}}}
\newcommand{\NormalTok}[1]{#1}
\usepackage{longtable,booktabs}
\usepackage{graphicx,grffile}
\makeatletter
\def\maxwidth{\ifdim\Gin@nat@width>\linewidth\linewidth\else\Gin@nat@width\fi}
\def\maxheight{\ifdim\Gin@nat@height>\textheight\textheight\else\Gin@nat@height\fi}
\makeatother
% Scale images if necessary, so that they will not overflow the page
% margins by default, and it is still possible to overwrite the defaults
% using explicit options in \includegraphics[width, height, ...]{}
\setkeys{Gin}{width=\maxwidth,height=\maxheight,keepaspectratio}
\IfFileExists{parskip.sty}{%
\usepackage{parskip}
}{% else
\setlength{\parindent}{0pt}
\setlength{\parskip}{6pt plus 2pt minus 1pt}
}
\setlength{\emergencystretch}{3em}  % prevent overfull lines
\providecommand{\tightlist}{%
  \setlength{\itemsep}{0pt}\setlength{\parskip}{0pt}}
\setcounter{secnumdepth}{5}
% Redefines (sub)paragraphs to behave more like sections
\ifx\paragraph\undefined\else
\let\oldparagraph\paragraph
\renewcommand{\paragraph}[1]{\oldparagraph{#1}\mbox{}}
\fi
\ifx\subparagraph\undefined\else
\let\oldsubparagraph\subparagraph
\renewcommand{\subparagraph}[1]{\oldsubparagraph{#1}\mbox{}}
\fi

%%% Use protect on footnotes to avoid problems with footnotes in titles
\let\rmarkdownfootnote\footnote%
\def\footnote{\protect\rmarkdownfootnote}

%%% Change title format to be more compact
\usepackage{titling}

% Create subtitle command for use in maketitle
\providecommand{\subtitle}[1]{
  \posttitle{
    \begin{center}\large#1\end{center}
    }
}

\setlength{\droptitle}{-2em}

  \title{respR}
    \pretitle{\vspace{\droptitle}\centering\huge}
  \posttitle{\par}
    \author{written by \textbf{Januar Harianto} and \textbf{Nicholas Carey}}
    \preauthor{\centering\large\emph}
  \postauthor{\par}
    \date{}
    \predate{}\postdate{}
  
\usepackage{booktabs}

\begin{document}
\maketitle

{
\setcounter{tocdepth}{1}
\tableofcontents
}
\chapter{Introduction}\label{introduction}

The \texttt{respR} package provides a structural, reproducible workflow
for the processing and analysis of respirometry data. While the focus of
our package is on aquatic respirometry, \texttt{respR} is largely
unitless, and so can process linear relationships in any time-series
data.

Most importantly, \texttt{respR} uses peer-reviewed methods for all of
its analyses, which means that all methods are well grounded in the
literature. View our acknowledgements section to view a list of methods
and their references.

Use \texttt{respR} to:

\begin{itemize}
\tightlist
\item
  automatically import raw data from various oxygen sensing equipment;
\item
  rapidly test data for issues before analysis;
\item
  explore and visualise timeseries data;
\item
  calculate linear segments of data manually or automatically;
\item
  convert units of oxygen consumption; and
\item
  export results quickly for reporting.
\end{itemize}

The goal of this guide is to walk you through the \texttt{respR} package
to import, analyse and convert your data. Before we begin, you will need
to \protect\hyperlink{installation}{install the package}.

\hypertarget{installation}{\chapter{Installation}\label{installation}}

We are finalising our plans to publish \texttt{respR} on CRAN. For now,
use the \texttt{devtools} package to install a stable version of the
package:

\begin{Shaded}
\begin{Highlighting}[]
\CommentTok{# install devtools}
\KeywordTok{install.packages}\NormalTok{(}\StringTok{"devtools"}\NormalTok{)}
\CommentTok{# use devtools to install respR from GitHub}
\NormalTok{devtools}\OperatorTok{::}\KeywordTok{install_github}\NormalTok{(}\StringTok{"januarharianto/respR"}\NormalTok{)}
\end{Highlighting}
\end{Shaded}

Next, load the \texttt{respR} package into your workspace:

\begin{Shaded}
\begin{Highlighting}[]
\KeywordTok{library}\NormalTok{(respR)}
\end{Highlighting}
\end{Shaded}

If you are using \texttt{respR} to analyse respirometry data for the
first time, check out our ``Getting Started'' guide. Otherwise, use the
Table of Contents to navigate your way to any topic of interest.

\chapter{Getting Started}\label{getting-started}

\section{Aquatic Respirometry}\label{aquatic-respirometry}

There are four broad methodological approaches in aquatic respirometry:
\emph{closed-chamber}, \emph{intermittent-flow}, \emph{flow-through} and
\emph{open-tank}.

In \textbf{closed-chamber} respirometry, \(O_2\) decrease is measured
within a hermetically sealed chamber of known volume, sometimes set
within a closed loop to allow mixing of the environment within the
chamber. Oxygen recordings may be continuous through use of an oxygen
probe, periodic through withdrawing water or gas samples at set
intervals, or a two-point measurement consisting of the initial and
final concentrations. Metabolic rates are estimated from the \(O_2\)
timeseries by assuming a linear relationship between variables, and
estimates of metabolic rate are straightforward in constant volume
respirometry using the equation: \[VO_2 = \dot O_2V\] where \(\dot O_2\)
is the slope of the regression that describes the rate of change in
\(O_2\) concentration over time, or in the case of a two-point
measurement, the difference in \(O_2\) concentration divided by time
elapsed, and \(V\) is the volume of fluid in the container (Lighton
2008).

In \textbf{intermittent-flow} respirometry, \(O_2\) concentration is
measured as described above, but periodically the chamber is flushed
with new water or air, returning it to initial conditions, resealed, and
the experiment repeated (Svendsen et al. 2016). This technique is
essentially the same as closed respirometry, but with the ability to
conduct replicates easily. Depending on the metabolic rate metric being
investigated, final respiration rate can be calculated as the mean of
the measures (e.g.~Carey et al. 2016), or the lowest or highest rates
recorded in any trial (e.g.~Stoffels 2015).

\textbf{Flow-through} respirometry involves a closed chamber, but with a
regulated flow of air or water through it at a precisely determined
rate. After equilibrium has been achieved, the oxygen concentration
differential between the incurrent and excurrent channels, along with
the flow rate, allows calculation of the oxygen extracted from the flow
volume per unit time: \[\dot{V}O_2 = (C_iO_2 - C_eO_2)FR\] where
\(\dot{V}O_2\) is the rate of \(O_2\) consumption over time, \(C_iO_2\)
and \(C_eO_2\) are the incurrent and excurrent \(O_2\) concentrations,
and \(FR\) is the flow rate through the system (Lighton 2008).

A final method is \textbf{open-tank} respirometry, in which a tank or
semi-enclosed area open to the atmosphere is used, but the input or
mixing rate of oxygen from the surroundings has been quantified or found
to be negligible relative to oxygen consumption of the specimens
(Leclercq et al. 1999). It is seldom used, but for some applications it
is a sufficient and practical methodology (Gamble et al. 2014). The
common equation used for open respirometry is:
\[\dot{V}O_2 = \dot O_2V + \phi_d\] where \(\dot O_2V\) is the slope of
the regression that relates \(O_2\) concentration to time, \(V\) is the
volume of the arena and \(\phi_d\) is the oxygen flux as determined by
Fick's Law (Leclercq et al. 1999).

\section{\texorpdfstring{The \texttt{respR} R
package}{The respR R package}}\label{the-respr-r-package}

\texttt{respR} is a package designed to process the data from all of
these types of respirometry experiment. It is designed primarily for
aquatic respirometry, although because many of the main functions are
unitless it is adaptable for use with gaseous respirometry, and indeed
analysis of other data where a parameter may change over time.

When working with respirometry data, you will often need to:

\begin{enumerate}
\def\labelenumi{\arabic{enumi}.}
\tightlist
\item
  Ensure that the data, or at least a \textbf{subset} of the data, is
  representative of the research question of interest.
\item
  Perform an initial analysis of the data to \textbf{estimate} the rate
  of change in oxygen concentration or amount.
\item
  Depending on the experimental setup, \textbf{correct} for background
  usage of oxygen by micro-organisms, or correct for oxygen flux from
  the air.
\item
  \textbf{Convert} the resulting usage rate to the volumetric and
  mass-specific rates in the appropriate units.
\end{enumerate}

The \texttt{respR} package allows determination of common respirometry
metrics and contains several functions to make this process
straightforward.

\begin{itemize}
\tightlist
\item
  It provides visual feedback and diagnostic plots to help you explore,
  subset and analyse your data.
\item
  It uses computational techniques such as \emph{rolling regressions}
  and \emph{kernel density estimates} to determine \textbf{maximum},
  \textbf{minimum} or \textbf{most linear} rates within time-series
  data.
\item
  The package takes an object-oriented approach, with all functions
  outputting objects which can be read by subsequent functions.
\item
  By separating the workflow into a series of connected functions, you
  can ``mix and match'' functions to help you achieve your result.
\item
  Output objects can also be saved or exported, and contain all raw
  data, parameters used in calculations, and results, allowing for a
  fully documented and reproducibile analysis of respirometry data.
\end{itemize}

\section{Example Data}\label{example-data}

We have provided example data that can be used immediately once
\texttt{respR} is loaded (\texttt{urchins.rd()},
\texttt{intermittent.rd()}, \texttt{zeb\_intermittent.rd()},
\texttt{sardine.rd()}, \texttt{squid.rd()}, \texttt{flowthrough.rd()}).

\begin{Shaded}
\begin{Highlighting}[]
\KeywordTok{data}\NormalTok{(}\DataTypeTok{package =} \StringTok{"respR"}\NormalTok{)}
\end{Highlighting}
\end{Shaded}

\chapter{Methods}\label{methods}

We describe our methods in this chapter.

\chapter{Applications}\label{applications}

Some \emph{significant} applications are demonstrated in this chapter.

\section{Example one}\label{example-one}

\section{Example two}\label{example-two}

\chapter{Final Words}\label{final-words}

We have finished a nice book.

\bibliography{book.bib,packages.bib}


\end{document}
