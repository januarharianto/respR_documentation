\documentclass[]{book}
\usepackage{lmodern}
\usepackage{amssymb,amsmath}
\usepackage{ifxetex,ifluatex}
\usepackage{fixltx2e} % provides \textsubscript
\ifnum 0\ifxetex 1\fi\ifluatex 1\fi=0 % if pdftex
  \usepackage[T1]{fontenc}
  \usepackage[utf8]{inputenc}
\else % if luatex or xelatex
  \ifxetex
    \usepackage{mathspec}
  \else
    \usepackage{fontspec}
  \fi
  \defaultfontfeatures{Ligatures=TeX,Scale=MatchLowercase}
\fi
% use upquote if available, for straight quotes in verbatim environments
\IfFileExists{upquote.sty}{\usepackage{upquote}}{}
% use microtype if available
\IfFileExists{microtype.sty}{%
\usepackage{microtype}
\UseMicrotypeSet[protrusion]{basicmath} % disable protrusion for tt fonts
}{}
\usepackage{hyperref}
\hypersetup{unicode=true,
            pdftitle={respR -- R Package Documentation},
            pdfauthor={Januar Harianto and Nicholas Carey},
            pdfborder={0 0 0},
            breaklinks=true}
\urlstyle{same}  % don't use monospace font for urls
\usepackage{natbib}
\bibliographystyle{apalike}
\usepackage{color}
\usepackage{fancyvrb}
\newcommand{\VerbBar}{|}
\newcommand{\VERB}{\Verb[commandchars=\\\{\}]}
\DefineVerbatimEnvironment{Highlighting}{Verbatim}{commandchars=\\\{\}}
% Add ',fontsize=\small' for more characters per line
\usepackage{framed}
\definecolor{shadecolor}{RGB}{248,248,248}
\newenvironment{Shaded}{\begin{snugshade}}{\end{snugshade}}
\newcommand{\KeywordTok}[1]{\textcolor[rgb]{0.13,0.29,0.53}{\textbf{#1}}}
\newcommand{\DataTypeTok}[1]{\textcolor[rgb]{0.13,0.29,0.53}{#1}}
\newcommand{\DecValTok}[1]{\textcolor[rgb]{0.00,0.00,0.81}{#1}}
\newcommand{\BaseNTok}[1]{\textcolor[rgb]{0.00,0.00,0.81}{#1}}
\newcommand{\FloatTok}[1]{\textcolor[rgb]{0.00,0.00,0.81}{#1}}
\newcommand{\ConstantTok}[1]{\textcolor[rgb]{0.00,0.00,0.00}{#1}}
\newcommand{\CharTok}[1]{\textcolor[rgb]{0.31,0.60,0.02}{#1}}
\newcommand{\SpecialCharTok}[1]{\textcolor[rgb]{0.00,0.00,0.00}{#1}}
\newcommand{\StringTok}[1]{\textcolor[rgb]{0.31,0.60,0.02}{#1}}
\newcommand{\VerbatimStringTok}[1]{\textcolor[rgb]{0.31,0.60,0.02}{#1}}
\newcommand{\SpecialStringTok}[1]{\textcolor[rgb]{0.31,0.60,0.02}{#1}}
\newcommand{\ImportTok}[1]{#1}
\newcommand{\CommentTok}[1]{\textcolor[rgb]{0.56,0.35,0.01}{\textit{#1}}}
\newcommand{\DocumentationTok}[1]{\textcolor[rgb]{0.56,0.35,0.01}{\textbf{\textit{#1}}}}
\newcommand{\AnnotationTok}[1]{\textcolor[rgb]{0.56,0.35,0.01}{\textbf{\textit{#1}}}}
\newcommand{\CommentVarTok}[1]{\textcolor[rgb]{0.56,0.35,0.01}{\textbf{\textit{#1}}}}
\newcommand{\OtherTok}[1]{\textcolor[rgb]{0.56,0.35,0.01}{#1}}
\newcommand{\FunctionTok}[1]{\textcolor[rgb]{0.00,0.00,0.00}{#1}}
\newcommand{\VariableTok}[1]{\textcolor[rgb]{0.00,0.00,0.00}{#1}}
\newcommand{\ControlFlowTok}[1]{\textcolor[rgb]{0.13,0.29,0.53}{\textbf{#1}}}
\newcommand{\OperatorTok}[1]{\textcolor[rgb]{0.81,0.36,0.00}{\textbf{#1}}}
\newcommand{\BuiltInTok}[1]{#1}
\newcommand{\ExtensionTok}[1]{#1}
\newcommand{\PreprocessorTok}[1]{\textcolor[rgb]{0.56,0.35,0.01}{\textit{#1}}}
\newcommand{\AttributeTok}[1]{\textcolor[rgb]{0.77,0.63,0.00}{#1}}
\newcommand{\RegionMarkerTok}[1]{#1}
\newcommand{\InformationTok}[1]{\textcolor[rgb]{0.56,0.35,0.01}{\textbf{\textit{#1}}}}
\newcommand{\WarningTok}[1]{\textcolor[rgb]{0.56,0.35,0.01}{\textbf{\textit{#1}}}}
\newcommand{\AlertTok}[1]{\textcolor[rgb]{0.94,0.16,0.16}{#1}}
\newcommand{\ErrorTok}[1]{\textcolor[rgb]{0.64,0.00,0.00}{\textbf{#1}}}
\newcommand{\NormalTok}[1]{#1}
\usepackage{longtable,booktabs}
\usepackage{graphicx,grffile}
\makeatletter
\def\maxwidth{\ifdim\Gin@nat@width>\linewidth\linewidth\else\Gin@nat@width\fi}
\def\maxheight{\ifdim\Gin@nat@height>\textheight\textheight\else\Gin@nat@height\fi}
\makeatother
% Scale images if necessary, so that they will not overflow the page
% margins by default, and it is still possible to overwrite the defaults
% using explicit options in \includegraphics[width, height, ...]{}
\setkeys{Gin}{width=\maxwidth,height=\maxheight,keepaspectratio}
\IfFileExists{parskip.sty}{%
\usepackage{parskip}
}{% else
\setlength{\parindent}{0pt}
\setlength{\parskip}{6pt plus 2pt minus 1pt}
}
\setlength{\emergencystretch}{3em}  % prevent overfull lines
\providecommand{\tightlist}{%
  \setlength{\itemsep}{0pt}\setlength{\parskip}{0pt}}
\setcounter{secnumdepth}{5}
% Redefines (sub)paragraphs to behave more like sections
\ifx\paragraph\undefined\else
\let\oldparagraph\paragraph
\renewcommand{\paragraph}[1]{\oldparagraph{#1}\mbox{}}
\fi
\ifx\subparagraph\undefined\else
\let\oldsubparagraph\subparagraph
\renewcommand{\subparagraph}[1]{\oldsubparagraph{#1}\mbox{}}
\fi

%%% Use protect on footnotes to avoid problems with footnotes in titles
\let\rmarkdownfootnote\footnote%
\def\footnote{\protect\rmarkdownfootnote}

%%% Change title format to be more compact
\usepackage{titling}

% Create subtitle command for use in maketitle
\providecommand{\subtitle}[1]{
  \posttitle{
    \begin{center}\large#1\end{center}
    }
}

\setlength{\droptitle}{-2em}

  \title{respR -- R Package Documentation}
    \pretitle{\vspace{\droptitle}\centering\huge}
  \posttitle{\par}
    \author{\textbf{Januar Harianto} and \textbf{Nicholas Carey}}
    \preauthor{\centering\large\emph}
  \postauthor{\par}
    \date{}
    \predate{}\postdate{}
  
\usepackage{booktabs}

\begin{document}
\maketitle

{
\setcounter{tocdepth}{1}
\tableofcontents
}
\chapter{Introduction}\label{introduction}

The R package \texttt{respR} provides a structural, reproducible
workflow for the processing and analysis of respirometry data. Although
the focus of the package is on aquatic respirometry, \texttt{respR} is
largely unitless, and so can process linear relationships in any
time-series data. All analytical methods used are peer-reviewed and
rigorously tested.

Use \texttt{respR} to:

\begin{itemize}
\tightlist
\item
  automatically import raw data from various oxygen sensing equipment;
\item
  rapidly test data for issues before analysis;
\item
  explore and visualise timeseries data;
\item
  calculate linear segments of data manually or automatically;
\item
  convert units of oxygen consumption; and
\item
  export results quickly for reporting.
\end{itemize}

The goal of this guide is to walk you through the \texttt{respR} package
to import, analyse and convert your data. Before we begin, you will need
to \protect\hyperlink{installation}{install the package}.

\chapter{\texorpdfstring{Why use
\texttt{respR}?}{Why use respR?}}\label{why-use-respr}

We have designed \texttt{respR} to be able to explore, process and
analyse \emph{any and all} aquatic respirometry data, independent of the
system used to collect it or type of experiment. Because of the unitless
nature of the majority of functions in the package, other respirometry
data (e.g.~aerobic), or other time series data examining other variables
can be explored and analysed in \texttt{respR}.

The use of simple data structures (numeric vector and data frames) means
there should be a low barrier to entry for anyone not completely new to
R. The unitless nature of the data (requiring only paired values of
numeric time-elapsed, and an oxygen amount in any unit) greatly reduces
the inputs each function requires, and simplifies any analysis using
\texttt{respR} in comparison to other packages (see
\href{https://januarharianto.github.io/respR/articles/packages_comp.html}{A
comparison of respR with other R packages}). Once the user has imported
and prepared their data to this form (see
\href{https://januarharianto.github.io/respR/articles/importing.html}{Importing
your data}), data analysis using respR is simple and intuitive.

\section{Other R respirometry
packages}\label{other-r-respirometry-packages}

Both \href{https://cran.r-project.org/web/packages/rMR/index.html}{rMR}
and \href{https://fishresp.org}{FishResp} centre around processing
intermittent-flow, swim tunnel respirometry data with multiple,
regularly-spaced replicates, particularly from Loligo Systems equipment.
If this does not describe your experiment there is little point in
considering these packages, as analysis of other respirometry
experiments in them is challenging, if not impossible.

\texttt{respR} is more than capable of processing these intermittent
flow experiments with a little forethought and data organisation
(\href{https://januarharianto.github.io/respR/articles/intermittent2.html}{see
here}), and support for this will get better in the coming months.
However, we would encourage users to explore these other packages and
how they work; they are useful options and may contain functionality
that suits your particular analyses or workflows better than
\texttt{respR}.

\section{Linear detection}\label{linear-detection}

Importantly, the packages mentioned above only allow manual selection of
data regions over which to determine rates, such as over a specified
time period. This is perfectly acceptable for many analyses, and
\texttt{respR} has this functionality (although in a much more flexible
implementation), but \texttt{respR} also has the \texttt{auto\_rate()}
function which identifies linear regions of respirometry data.

This powerful function allows identification of \emph{most linear,
minimum and maximum} rates in an \textbf{independent, objective, and
statistically robust manner}. We would encourage the respirometry
community to explore this objective method rather than rely on manual
selection, which can leave investigators open to accusations of cherry
picking and bias when reporting metabolic rates.

Another R package, \texttt{LoLinR}, can identify linear sections in time
series data, athough using a fundamentally different method than
\texttt{auto\_rate()}. However, \texttt{LoLinR} is extremely
computationally inefficient, taking literally hours to days to process
typical respirometry datasets. It does appear to perform well, even
after subsampling longer data to the shorter lengths the function can
handle, but \texttt{auto\_rate}appears to perform equally well and can
process these data in seconds without modification (see
\href{https://januarharianto.github.io/respR/articles/auto_rate_comp.html}{here}).

\section{Reproducibility}\label{reproducibility}

Our other main objective with \texttt{respR} was to provide a solution
for reporting analyses of respirometry data in an easily reproducible
form. See
\href{https://januarharianto.github.io/respR/articles/reproducibility.html}{Open
science and reproducibility using respR} for an example, but in summary,
an entire \texttt{respR} analysis can be reported in only a few lines of
code. Our careful selection of descriptive function names and input
operators allow - we hope - this code to be readable and convey easy
understanding about what is being done in the analysis, even without
additional comments. Inclusion of a raw data file with this code would
allow anyone to reproduce it easily, and scrutinise each stage. We hope
this makes the job of investigators, reviewers and editors easier.

\section{Using other similar R
packages}\label{using-other-similar-r-packages}

We see only a few scenarios in which other R packages might be
considered for analysing respirometry data:

\begin{itemize}
\item
  \texttt{LoLinR} - If your data is shorter than around 500 datapoints,
  and you would like to use a different linear detection method other
  than \texttt{auto\_rate} in \texttt{respR}, \texttt{LoLinR} can
  achieve this. Note however, this package is not respirometry focussed.
  You would still need to utilise \texttt{respR} or other solutions to
  format and import respirometry data, apply background corrections, and
  convert the resulting slopes to particular units. If your data are
  longer than around 400-500 in length, \texttt{LoLinR} is not a
  practical option because of the time needed to process the data.
\item
  \texttt{rMR}/\texttt{FishResp} - If you are doing intermittent-flow,
  swim tunnel experiments with regularly spaced replicates (particularly
  using Loligo Systems equipment) these are good options for analysising
  your data. These data are also able to be analysed in \texttt{respR}
  however: see
  \href{https://januarharianto.github.io/respR/articles/intermittent2.html}{link}.
\item
  Gas/Air respirometry - We don't know of any options in R to analyse
  air respirometry experiments, and at present \texttt{respR} is chiefly
  designed to analyse aquatic respirometry data. However, the great
  majority of the exploratory and analytical functions are unit
  agnostic. The only step requiring units is conversion of slopes to
  rates in the function \texttt{convert\_rate()}. While we have not done
  this, we imagine air respirometry data could easily be explored and
  rates determined, and most experienced investigators in air
  respirometry would be able to convert these to units themselves. We
  plan to add this support, but as aquatic biologists we have no
  experience with these data. Please get in
  \href{mailto:nicholascarey@gmail.com}{touch} if you can help with
  this.
\end{itemize}

\section{\texorpdfstring{Using
\texttt{respR}}{Using respR}}\label{using-respr}

In our opinion, you should use \texttt{respR} for most of your aquatic
respirometry data.

We have built \texttt{respR} to accept any and all types of experiment
or data source easily. We have not processed every data format or
variation of experiment, and there may be circumstances we have not
anticipated. So, if for whatever reason some part of \texttt{respR} does
not work for you please do \href{mailto:nicholascarey@gmail.com}{let us
know}, and we will work to accommodate it.

\part*{USAGE}\label{part-usage}
\addcontentsline{toc}{part}{USAGE}

\hypertarget{installation}{\chapter{Installation}\label{installation}}

Use the \texttt{devtools} package to install a stable version of
\texttt{respR}:

\begin{Shaded}
\begin{Highlighting}[]
\CommentTok{# install devtools}
\KeywordTok{install.packages}\NormalTok{(}\StringTok{"devtools"}\NormalTok{)}
\CommentTok{# use devtools to install respR from GitHub}
\NormalTok{devtools}\OperatorTok{::}\KeywordTok{install_github}\NormalTok{(}\StringTok{"januarharianto/respR"}\NormalTok{)}
\end{Highlighting}
\end{Shaded}

The developmental version is mostly stable, but contains bleeding-edge
improvements. Do not use it unless you need new functionality that is
still being tested in this branch. You can install the package using the
code below:

\begin{Shaded}
\begin{Highlighting}[]
\CommentTok{# install dev version}
\NormalTok{devtools}\OperatorTok{::}\KeywordTok{install_github}\NormalTok{(}\StringTok{"januarharianto/respR"}\NormalTok{, }\DataTypeTok{ref =} \StringTok{"develop"}\NormalTok{)}
\end{Highlighting}
\end{Shaded}

Next, load the \texttt{respR} package into your workspace:

\begin{Shaded}
\begin{Highlighting}[]
\KeywordTok{library}\NormalTok{(respR)}
\end{Highlighting}
\end{Shaded}

Check out our \protect\hyperlink{quick-start}{Quick start} guide if you
are using \texttt{respR} for the first time.

\hypertarget{quick-start}{\chapter{Quick start}\label{quick-start}}

\texttt{respR} has been designed to be simple to use, even for novice R
users. Here we show you some examples of how data is analysed using
various functions of the package.

\section{Example data}\label{example-data}

Example data are available when \texttt{respR} is loaded:
\texttt{urchins.rd()}, \texttt{intermittent.rd()},
\texttt{zeb\_intermittent.rd()}, \texttt{sardine.rd()},
\texttt{squid.rd()}, \texttt{flowthrough.rd()}. Because lazy loading is
implemented, the data can be called immediately. To view the list of
available datasets and their descriptions, run the code:

\begin{Shaded}
\begin{Highlighting}[]
\KeywordTok{data}\NormalTok{(}\DataTypeTok{package =} \StringTok{"respR"}\NormalTok{)}
\end{Highlighting}
\end{Shaded}

\section{Demonstration}\label{demonstration}

Let's analyse a relatively complex dataset quickly using \texttt{respR}.

\part*{METHODS}\label{part-methods}
\addcontentsline{toc}{part}{METHODS}

\chapter{Aquatic respirometry}\label{aquatic-respirometry}

There are four broad methodological approaches in aquatic respirometry:
\emph{closed-chamber}, \emph{intermittent-flow}, \emph{flow-through} and
\emph{open-tank}.

\section{Closed-chamber}\label{closed-chamber}

\begin{quote}
\begin{itemize}
\tightlist
\item
  Use \texttt{calc\_rate()} and \texttt{auto\_rate()} to measure
  closed-chamber respirometry.
\end{itemize}
\end{quote}

In \textbf{closed-chamber} respirometry, \(O_2\) decrease is measured
within a hermetically sealed chamber of known volume, sometimes set
within a closed loop to allow mixing of the environment within the
chamber. Oxygen recordings may be continuous through use of an oxygen
probe, periodic through withdrawing water or gas samples at set
intervals, or a two-point measurement consisting of the initial and
final concentrations. Metabolic rates are estimated from the \(O_2\)
timeseries by assuming a linear relationship between variables, and
estimates of metabolic rate are straightforward in constant volume
respirometry using the equation: \[VO_2 = \dot O_2V\] where \(\dot O_2\)
is the slope of the regression that describes the rate of change in
\(O_2\) concentration over time, or in the case of a two-point
measurement, the difference in \(O_2\) concentration divided by time
elapsed, and \(V\) is the volume of fluid in the container (Lighton
2008).

\section{Intermittent}\label{intermittent}

\begin{quote}
\begin{itemize}
\tightlist
\item
  Use \texttt{calc\_rate()} to manually measure intermittent
  respirometry.
\item
  Use \texttt{auto\_rate()} if segments are manually identified prior to
  running the analysis.
\end{itemize}
\end{quote}

In \textbf{intermittent-flow} respirometry, \(O_2\) concentration is
measured as described above, but periodically the chamber is flushed
with new water or air, returning it to initial conditions, resealed, and
the experiment repeated (Svendsen et al. 2016). This technique is
essentially the same as closed respirometry, but with the ability to
conduct replicates easily. Depending on the metabolic rate metric being
investigated, final respiration rate can be calculated as the mean of
the measures (e.g.~Carey et al. 2016), or the lowest or highest rates
recorded in any trial (e.g.~Stoffels 2015).

\section{Flow-through}\label{flow-through}

\begin{quote}
\begin{itemize}
\tightlist
\item
  Use \texttt{calc\_rate.ft()} to measure flow-through respirometry.
\end{itemize}
\end{quote}

\textbf{Flow-through} respirometry involves a closed chamber, but with a
regulated flow of air or water through it at a precisely determined
rate. After equilibrium has been achieved, the oxygen concentration
differential between the incurrent and excurrent channels, along with
the flow rate, allows calculation of the oxygen extracted from the flow
volume per unit time: \[\dot{V}O_2 = (C_iO_2 - C_eO_2)FR\] where
\(\dot{V}O_2\) is the rate of \(O_2\) consumption over time, \(C_iO_2\)
and \(C_eO_2\) are the incurrent and excurrent \(O_2\) concentrations,
and \(FR\) is the flow rate through the system (Lighton 2008).

\section{Open-tank}\label{open-tank}

\begin{quote}
\begin{itemize}
\tightlist
\item
  Use \texttt{calc.rate()} and \texttt{auto\_rate()} to measure
  open-tank respirometry, and then use \texttt{adjust\_rate()} to
  include oxygen flux adjustments.
\end{itemize}
\end{quote}

A final method is \textbf{open-tank} respirometry, in which a tank or
semi-enclosed area open to the atmosphere is used, but the input or
mixing rate of oxygen from the surroundings has been quantified or found
to be negligible relative to oxygen consumption of the specimens
(Leclercq et al. 1999). It is seldom used, but for some applications it
is a sufficient and practical methodology (Gamble et al. 2014). The
common equation used for open respirometry is:
\[\dot{V}O_2 = \dot O_2V + \phi_d\] where \(\dot O_2V\) is the slope of
the regression that relates \(O_2\) concentration to time, \(V\) is the
volume of the arena and \(\phi_d\) is the oxygen flux as determined by
Fick's Law (Leclercq et al. 1999).

\part*{EXAMPLES}\label{part-examples}
\addcontentsline{toc}{part}{EXAMPLES}

\chapter{Intermittent respirometry}\label{intermittent-respirometry}

Some \emph{significant} applications are demonstrated in this chapter.

\section{Example one}\label{example-one}

\section{Example two}\label{example-two}

\part*{ARTICLES}\label{part-articles}
\addcontentsline{toc}{part}{ARTICLES}

\chapter{Articles}\label{articles}

\bibliography{book.bib,packages.bib}


\end{document}
